\chapter{Abstract}

The aim of the current thesis consists in the addition of new functionalities to a pothole detection system in order to capture tridimensional depressions on roads. In addition to that, It will be exposed different managing pothole software tools as well as hardware and software regarding the measure of them. Moreover several techniques for pothole processing and clasification will be mentioned.

As regards pothole managing software, It will be briefly presented concepts related to different sorts of depressions on roads along with former software researches about pothole measuring and managing on information systems.

As of hardware devices, It will be studied diverse kind of surface sensing devices including its main features and technique used for surface representation. Moreover programming languages, libraries, tools and availability will be mentioned. Furthermore gps-compatible software utilities and web libraries which allow to get address approximate information from gps will be showed.

Besides techniques involving filtering and classification of potholes will be detailed together with former researches and similar projects.

Finally, additions to an existing web application previously developed will be done with the objective of including visualization of various types of pavement depressions on a webbrowser plus filtering of failures by street name and address computing from latitude and longitude of potholes. Apart from that, a client aplication for different sorts of potholes with the Kinect device will be developed which will be able to retrieve and send information from the web application concerning the location of the pavement depressions formerly registered on the web application by users. Additionally, pothole data files will be available for visualization on the web application.
What is more, the client application will allow local user to register potholes, which are not signed up on the web application, spotted during the patrol and getting the latitude and longitude by a GPS device for future estimation on the web application. Then, the user will be capable of selecting desired pothole files and uploading them from this application.

Lastly, an independent script, which makes use of diverse machine learning filters and algorithms given by Point Cloud Library (PCL), will be developed in order to attain the isolation and classification of Potholes and Cracks pavement depressions. 




