%Cambiar el formato de las secciones, subsecciones y subsubsecciones
\definecolor{miazul}{RGB}{0,82,155}
\titleformat{\section}[block]{\Large\bfseries\sffamily\color{miazul}} 
              {\thesection.}{1pt}{}

\titleformat{\subsection}[block]{\Large\bfseries\sffamily\color{miazul}} 
              {\thesubsection.}{1pt}{}

\titleformat{\subsubsection}[block]{\Large\bfseries\sffamily\color{miazul}} 
              {\thesubsubsection.}{1pt}{}


\chapter*{Organización de los contenidos}
%\subsection{Output formats}

\section*{Capítulo 1.Introducción}

En este capítulo se introducen los objetivos de la tesina, el marco teórico y los desarrollos propuestos.

\section*{Capítulo 2.Antecedentes de software para la gestión de fallas viales}

Este capítulo comienza definiendo el concepto de falla, los tipos de material, tipos de reparación que se pueden efectuar sobre cada uno, y los tipos de falla que se corresponden sobre los distintos tipos de material. También se debaten antecedentes históricos, software y proyectos de investigación, relativos al sensado de fallas en distintos contextos.

\section*{Capítulo 3.Dispositivos hardware y herramientas software para el sensado de fallas}

El objetivo de este capítulo es enumerar los distintos tipos de dispositivos y sensores que pueden utilizarse para realizar la captura de irregularidades viales, sus principales características y funcionamiento. 
Por otro lado, se expondrán librerías en distintos lenguajes de programación, y métodos de almacenamiento de las fallas empleados por los mismos.

\section*{Capítulo 4.Técnicas de reconocimiento y procesamiento de fallas}

Durante el desarrollo de este capítulo se definirá el concepto de Machine Learning, sus usos, contextos de aplicación y algoritmos que brinda para el reconocimiento de patrones en conjuntos datos. Posteriormente, se expondrán aquellas técnicas empleadas para realizar la identificación, el saneamiento y la clasificación de los datos de una nube de puntos con datos de una falla.

\section*{Capítulo 5. Herramientas GPS y Geocoding}

El objetivo de este capítulo consiste en explicar el concepto de GPS, latitud, longitud, funcionamiento, dispositivos GPS disponibles y utilidades software para compartir conexiones GPS entre sistemas operativos Android y Linux. Además, se explicará el concepto de Geocoding y Reverse Geocoding, librerías disponibles en PHP y Python y, servicios web ofrecidos por terceros compatibles con éstas.

\section*{Capítulo 6. Caso de aplicación}

El objetivo de este capítulo es explicar tanto la arquitectura general de la aplicación cliente como de la aplicación web, los objetivos de las mismas, el ciclo de trabajo y las formas de interacción con ambas aplicaciones. 
Por un lado, se expondrá el funcionamiento de la aplicación cliente, librerías empleadas para su desarrollo, formas de interacción de ésta con el sensor Kinect y diseño.
Por otro lado, se presentará la funcionalidad base que la aplicación web ofrecía anteriormente, diseño, la funcionalidad incluida como parte del desarrollo de la presente investigación y librerías relacionadas al funcionamiento agregado a la misma. 

\section*{Capítulo 7. Conclusiones y líneas futuras}

En este capítulo, se explican las conclusiones de la investigación realizada y posibles líneas de investigación futuras con el presente desarrollo.